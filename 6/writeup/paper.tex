\section{Paper review: <Roofline: An Insightful Visual Performance Model for Multicore Architectures>}
The paper "Roofline: An Insightful Visual Performance Model for Multicore Architectures" published by S. Williams, A. Waterman and D. Patterson in 2009 introduces the simplistic "Roofline" model with the aim to identify performance boundaries in multicore systems. It separates the operational intensity (Flops/DRAM byte accessed) of a task into two domains: Low operational intensity, where peak memory bandwidth is limiting, and high operational intensity, where instead the peak floating point performance of the processor is the bottleneck of the computation. The model delivers a universal upper baseline which developers can use to orientate on when optimizing code for a given processing design, since the Roofline model only depends on the architecture.\\
To better account for situations way below the upper boundary of performance, the authors also propose a set of ceilings which mark the inclusion of different computing and memory optimizations, which include instruction level parallelism, single-instruction-multiple-data, a balanced floating point operation mix, adjacent memory loop design, memory affinity ensuring, and software prefetching. They propose to employ a number of "microbenchmarks" in order to measure the position of the ceilings, which then can be used to select a promising optimization for a given operational intensity, where the potential gains are given by the distance between two ceilings. Since the model implicitly includes cache optimizations as increases in operational intensity, the authors describe the possibility to include the "3c" cache miss model to further improve the models predictability.\\
The Roofline model oversimplifies the performance behavior of processors, e.g assuming operational intensity per kernel, in order to make a easy to handle model for developers, and is still in use ten years later. Its effectiveness and flexibility are shown on four different architectures, demonstrating the different regimes of performance enhancement, depending on the operational intensity of the used kernels. In the scope of this lecture, the Roofline model is also relevant for GPU computing, delivering a simple and reliable metric for performance optimization.




